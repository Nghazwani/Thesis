% example of intro , Dr.Ken helped with %

The standard model (SM) of particle physics %successfully
is the best known theoretical model that describes interactions of matter at the smallest %space and time
distance scales and it describes a vast range of subatomic phenomena with outstanding precision.Over the period of its development starting in the latter half of the twentieth century, the predictive power of the SM theory has been validated through the discovery of the elementary particles such as the Z and W bosons in 1983 at CERN's Super Proton-Antiproton Synchrotron, and the top quark in 1995 at the Tevatron located at the Fermi National Accelerator Laboratory. The most notable and last confirmation of the SM was the joint discovery of the Higgs boson in 2012 at CERN's Large Hadron Collider (LHC) by the CMS and ATLAS Collaborations.

Nonetheless, the SM has well-known shortcomings.Several observed phenomena, such as the existence of dark matter, massive neutrinos, and the matter-antimatter asymmetry, cannot be explained through the SM alone.
These facts motivate further precision measurements of the physical properties and interactions of the SM particles, studies of rare processes predicted the SM, and explicit searches for physics beyond the SM.

Many of these studies have taken place at the world's largest and most energetic particle accelerator, the LHC, which began operation in 2009. The LHC collides two proton beams at a center-of-mass energy up to $\sqrt{s}=13.6$ TeV at several points along the collider.Two of the interaction points are at the center of large general purpose particle detectors, CMS and ATLAS. My research was conducted with the CMS experiment at the LHC.

Physics analyses at the LHC are enabled with identifying particles produced in each proton-proton collision with the best estimate of their properties including 4-momenta.
This task of ``particle reconstruction'' is carried out by the particle flow (PF) algorithm, which uses information from all of the subdetectors
to reconstruct candidates (PF candidates) of charged hadrons, neutral hadrons, photons, electrons, and muons.
This PF algorithm is the main topic of my dissertation.
More specifically, my research was on the calibration of clusters formed from energy deposits in the CMS calorimeters.

% outline of the thesis %

%This thesis is organized as follows. Chapter ...
The work in this thesis is organized as follows:
%chapter 2%
Chapter Two presents an overview of the standard model theory.
%chapter 3%
Next, Chapter Three describes briefly the Large Hadron Collider, the CMS detector and its main components.
%chapter 4%
The concept of the PF used in event reconstruction is introduced in Chapter Four with a focus on calorimeter cluster calibration, which is related to work in this thesis.
%chapter 5%
Both chapters Five and Six outline the details of performing the PF cluster calibration for both ECAL and HCAL on Run~3 datasets samples.
%outline the utilized datasets, ML models used in performaing the PF cluster calibration for both ECAL and HCAL.
%chapter 6%
Lastly, Chapter Seven includes a summary of the results.
