% intro CMS %
The Compact Muon Solenoid (CMS) (source) is a general-purpose detector at the LHC. The CMS physics program ranges from studying the SM, including the Higgs boson, to searching for extra dimensions and particles that could make up dark matter. The design of CMS and its subdetectors is based on the geometry of its solenoid magnet, as shown in (fig). 
 
The following sections describe the different subdetectors in the CMS experiment:     


\subsection{Superconducting Magnet} 
The CMS magnet is a superconducting magnet that produces a field 100,000 times stronger than the Earth's (source). This magnetic field is used to determine the momenta of charged particles since the particles' trajectories bend in the field.

\subsection{Inner Tracker} 
The first subdetector the particles will interact with after leaving the beam pipe is the tracker. The main purpose of this detector is to track the charged particles' trajectory to find their momentum. The tracker has two parts: the pixel detector (Source), which has silicon sensors (pixels) in four layers, and the silicon microstrip detector (source), which surrounds the pixels and ten layers of silicon strips. Pictures of the inner tracker parts are shown in (fig:)

\subsection{Electromagnetic Calorimeter}
The CMS measures the energies of the emerging particles using calorimeters. The electromagnetic Calorimeter (ECAL) (source) measures the energy of electrons and photons. The ECAL is made of lead tungstate crystals, which are heavier than stainless steel but transparent. These crystals scintillate when electrons and photons pass through them, producing light proportional to the particle energy. The ECAL comprises a barrel section and two endcaps, as shown in (fig). For more precision, the ECAL has an extra detector in front of the endcaps, which is called a preshower.

\subsection{Hadronic Calorimeter}
The Hadronic Calorimeter (HCAL) (source) measures directly the energy of hadrons, such as neutrons and pions, and indirectly the presence of non-interacting particles like neutrinos. The HCAL is a sampling calorimeter made of alternating layers of absorber, dense material like brass or steel, and scintillator, plastic tiles that produce a rapid light pulse when the particle passes through. The HCAL is massive and thick to capture the cascades of particles produced when a hadron hits the absorber materials. The HCAL sections, as shown in this (fig), are: barrel (HB) and outer barrel (HO), endcap (HE), and forward (HF). 

\subsection{Muon Detector}
Unlike most particles, muons are not stopped by any of CMS’s calorimeters, so the muon chambers are placed at the very edge of the CMS detector. The muon system (source) has four muon stations (MS), which sit outside the magnet coil and are interleaved with iron return yoke plates. As shown in (fig), The MSs in the barrel region are drift Tubes (DTs) and Cathode Strip Chambers (CSCs). In the endcap region, we have CSCs, resistive plate chambers (RPCs), and gas electron multiplier chambers (GEMs), which will be completely installed before phase 2 of the LHC. (source)   


% figures % 
