% intro CMS %
The Compact Muon Solenoid (CMS) (source) is a general-purpose detector at the LHC. The CMS physics program spans over a wide range of studies from exploring the processes predicted by the Standard Model to searching for extra dimensions and particles that could make up the dark matter. The design of CMS and its subdetectors is based on the geometry of its solenoid magnet, as shown in (fig).

The following sections describe the different subdetectors in the CMS experiment:


\subsection{Superconducting Magnet}
The CMS magnet is a superconducting magnet that produces a field 100,000 times stronger than the Earth's (source). This strong magnetic field is essential for measuring the momenta of charged particles as the curvature of their trajectories is directly related to their momentum.

\subsection{Inner Tracker}
%KenH The first subdetector the particles will interact with after leaving the beam pipe is the tracker.
The tracker is the first subdetector that particles encounter after leaving the beam pipe.
The main purpose of this detector is to track the charged particles' trajectory in order to find their momentum.
The tracker has two parts: the pixel detector (source), which has silicon sensors (pixels) in four layers, and the silicon microstrip detector (source), which surrounds the pixels and ten layers of silicon strips. Pictures of the inner tracker parts are shown in (fig:).

\subsection{Electromagnetic Calorimeter}
%KenH The CMS measures the energies of the emerging particles using calorimeters.
Calorimeters are essential subdetectors designed to measure the energies of the majority of incident particles.
The Electromagnetic Calorimeter (ECAL) (source) measures the energy of electrons and photons.
The CMS ECAL is made of lead tungstate crystals, which are heavier than stainless steel but transparent.
These crystals scintillate when electrons and photons pass through them, producing light proportional to the particle energy.
The ECAL comprises a barrel section and two endcaps, as shown in (fig).
For a better separation between prompt photons and those originating from light meson decays, the ECAL has an extra detector in front of the endcaps, which is called a preshower detector.

\subsection{Hadronic Calorimeter}
The Hadronic Calorimeter (HCAL) (source) measures directly the energy of hadrons, such as neutrons and pions, and it is also essential for indirectly detecting the presence of non-interacting particles such as neutrinos.
The HCAL is a sampling calorimeter made of alternating layers of absorber, dense material like brass or steel, and scintillator, plastic tiles that produce a rapid light pulse when the particle passes through.
The HCAL is massive and thick to capture the cascades of particles produced when a hadron hits the absorber materials.
The HCAL sections, as shown in this (fig), are: barrel (HB) and outer barrel (HO), endcap (HE), and forward (HF).

\subsection{Muon Detector}
Unlike most particles, muons are not absorbed by any of CMS’s calorimeters, so the muon chambers are placed at the outer edge of the CMS detector.
The muon system (source) has four muon stations, which sit outside the magnet coil and are interleaved with iron return yoke plates as shown in (fig).
The muon stations in the barrel region consists of Drift Tubes (DTs) and resistive plate chambers (RPCs).
In the endcap region, there are Cathode Strip Chambers (CSCs), RPCs, and gas electron multiplier chambers (GEMs), whose installation will be completed before the Phase 2 operation of the LHC. (source)

% figures %
