The calibration of calorimeter clusters serves as a crucial step in the particle flow algorithm, potentially enhancing particle identification and reducing the likelihood of misreconstructed energy in the CMS experiment at the LHC. With this thesis, we showed that ML techniques offline/online PF cluster calibration can done with high response/ resolution. First, we tested the performance of the ML model derived from an earlier run on 2024 data-taking decisions. These studies showed that the existing correction works well; therefore, the same ML models were used for the 2024 PF ECAL calibration. Next, we compared the offline and online versions of PF clusters in terms of energy response, which showed that the ML-based calibration obtained from the offline clusters can also be applied to online clusters. Lastly, we checked the calibration done by the DRN and compared it to the traditional method. The DRN model generally provides good response, except for the endcap response outside the tracker, where the energy response becomes unstable. However, a significant improvement in energy resolution was observed for EH hadrons. 






%%%%%%%%%%%%%%%%%%%%%%%%%%%%%%%%%%%%%%%%%%%%%%%%%%%%%%%
%\section{Lists}

%Lists have several characteristics:

%\begin{itemize}
%\item Lists can contain items
%\item Lists can contain other lists
%\begin{itemize}
%\item Different depths have different bullets
%\item Different depths have different margins
%\end{itemize}
%\item Lists are easy
%\end{itemize}

%\section{Enumerations}

%A reasonable chapter layout for the thesis might be

%\begin{enumerate}
%\item Introduction
%\item Related Work
%\item Methodology
%\item Results and Analysis
%\item Conclusion
%\end{enumerate}

