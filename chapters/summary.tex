
The Compact Muon Solenoid (CMS) is a general-purpose experiment at the Large Hadron Collider (LHC) designed to study a wide range of physics phenomena.
The majority of physics analyses from the CMS experiment relies on the particle flow algorithm to reconstruct particle candidates produced in proton-proton collisions.
%
Calibrating the calorimeter clusters is a crucial step in the particle flow algorithm, which enhances particle identification and reduces the likelihood of misreconstructed energy in the CMS experiment at the LHC.

This thesis comprises two main components.
%The work in this thesis is organized into two sections.
The first part concerns with the calibraiton of clusters formed by energy deposits in the electromagnetic calorimeter (ECAL).
For this PF ECAL cluster calibration,
a machine-learning (ML) model, based on boosted decision tree, derived from an earlier run was applied to the PF ECAL cluster calibration sample simulated under the 2024 data-taking condition.
The results showed that the existing calibration works well; therefore, the same ML models were used for the 2024 PF ECAL calibration.
Also, a comparison between the offline and online versions of PF ECAL clusters was performed, and the results showed that the ML-based calibration obtained from offline clusters can be applied to online clusters.

Another ML method, Dynamic Reduction Network (DRN), was tested and compared to the conventional method for hadron cluster calibration in the second part.
The model generally provides a good energy response, except for the endcap outside the tracker covorage region, where the energy response becomes unstable.
However, a promising improvement in energy resolution has been observed for hadrons, especially for those called EH hadrons which start showers in the ECAL before reaching the hadron calorimeter.

%%%%%%%%%%%%%%%%%%%%%%%%%%%%%%%%%%%%%%%%%%%%%%%%%%%%%%%
%\section{Lists}

%Lists have several characteristics:

%\begin{itemize}
%\item Lists can contain items
%\item Lists can contain other lists
%\begin{itemize}
%\item Different depths have different bullets
%\item Different depths have different margins
%\end{itemize}
%\item Lists are easy
%\end{itemize}

%\section{Enumerations}

%A reasonable chapter layout for the thesis might be

%\begin{enumerate}
%\item Introduction
%\item Related Work
%\item Methodology
%\item Results and Analysis
%\item Conclusion
%\end{enumerate}

