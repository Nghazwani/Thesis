Calibrating the calorimeter clusters is a crucial step in the particle flow algorithm, which enhances particle identification and reduces the likelihood of misreconstructed energy in the CMS experiment at the LHC.
The work in this thesis is organized into two sections.
In the first part, the performance of the ML (BDT) model derived from an earlier run was applied to 2024 data-taking.
The results showed that the existing calibration works well; therefore, the same ML models were used for the 2024 PF ECAL calibration.
Another ML method (DRN) was tested and compared to the conventional method for hadron cluster calibration in the second part.
The model generally provides a good energy response, except for the endcap outside the tracker, where the energy response becomes unstable.
However, a significant improvement in energy resolution was observed for EH hadrons.
Also, a comparison between the offline and online versions of PF HCAL clusters was studied, and the results showed that the ML-based calibration obtained from offline clusters can be applied to online clusters.






%%%%%%%%%%%%%%%%%%%%%%%%%%%%%%%%%%%%%%%%%%%%%%%%%%%%%%%
%\section{Lists}

%Lists have several characteristics:

%\begin{itemize}
%\item Lists can contain items
%\item Lists can contain other lists
%\begin{itemize}
%\item Different depths have different bullets
%\item Different depths have different margins
%\end{itemize}
%\item Lists are easy
%\end{itemize}

%\section{Enumerations}

%A reasonable chapter layout for the thesis might be

%\begin{enumerate}
%\item Introduction
%\item Related Work
%\item Methodology
%\item Results and Analysis
%\item Conclusion
%\end{enumerate}

