%% Ch4:Hadronic Cluster Calibration UsinAg Graph Neural Network%% 

\section{Introduction to Graph Neural Network}
GNN is a type of NN that is used to process data that can be represented as graphs.
% https://en.wikipedia.org/wiki/Graph_neural_network
% check: A Gentle Introduction to Graph Neural Networks https://distill.pub/2021/gnn-intro/
% check: Basic Understanding of Neural Network Structure https://medium.com/@sarita_68521/basic-understanding-of-neural-network-structure-eecc8f149a23
% check: Epochs, Batch Size, Iterations https://www.sabrepc.com/blog/Deep-Learning-and-AI/Epochs-Batch-Size-Iterations
%=================================================

Where GNN's are used in CMS?
- For Calorimetry, In both ECAL & HCAL.

% what are NN?
% what are GNN?

 where GNN can be used in ECAL?
- for superclustering "Particle reconstruction", particle identification (identify the flavor of the particle), energy regression. 

when do we perform energy correction?
- after superclusters are formed, energy corrections are applied.
- this correction account for the energy lost in gaps and upstream material .. etc
-  currently this calibration is done using BDT with semiparametric regression. (this work is covered in previous chapter)
-  new ML algorithm developed DRN (to use low level inputs (rechits) instead of high level inputs (describe EM showers))

Energy corrections using Dynamic Reduction Network?
- input features (rechits) are transformed into high dimensional latent space.
-  Then graphs are dynamically generated in the latent space.
-  the graph convolution are performed (includes message passing)
-  the info is aggregated over the graph using clustering and pooling.
 => input: Rechit features (E,x,y,z) => output: Predict probability density of energy correction value.
%% slide 16 has resolution plot that compare BDT to DRN calibration. (using photon gun simulation with ideal detector calibration)

% source: ML for calorimetry by Polina simkina

%================================================

DRN architecture overview:
%..........................
%% 1. input: Rechits => inputNet (FCNN) *Fully Convolutional Network?*
%% 2. => Graph Generation (KNN) => EdgeConv => calculate edge weights => graph clustering (Graclus) => graph pooling (add)
%% 3. Global pool (max)
%% 4. outputNet => output (E pred)

Training target?
%% minimize loss - [(E true- E pre)^/E true]

 DRN training parapmeters?
%% Input layers: 3
%% aggregation layers: 2
%=> *An aggregation layer is a crucial part of network architecture that connects the access layer to the core layer.
%=> It's also known as the distribution layer.
%=> The aggregation layer's main purpose is to increase network scalability and manage traffic*
%=> source: https://www.sciencedirect.com/topics/computer-science/aggregation-layer#:~:text=The%20aggregation%20layer%20is%20defined,with%20vPC%20in%20computer%20networks.
%% message passing layers:3
%=> *Message passing layers are permutation-equivariant layers mapping a graph into an updated representation of the same graph.
%=> Formally, they can be expressed as message passing neural networks (MPNNs)* source: https://en.wikipedia.org/wiki/Graph_neural_network#Message_passing_layers
%% output layers: 2, *why 2 instead of 1??*
%% hidden dimension: 64
%% batch size: 100
%=> Batch Size is among the important hyperparameters in Machine Learning.
%=> It is the hyperparameter that defines the number of samples to work through before updating the internal model parameters.
%=> source: https://medium.com/geekculture/how-does-batch-size-impact-your-model-learning-2dd34d9fb1fa
%% optimizer w/ constant learning reate of 0.001 AdamW
%=> check video https://www.youtube.com/watch?v=oWZbcq_figk to learn more about AdamW
%% model parameters: 62405
%% total events # training, # validation. 

% Rechits (input): slide 11
%.................
% where do we get the PF Rechits? 
%% for each event we get the desired PF candidate. then we get the corresponding elements in its PFBlock (tracks/cluster).
%% then we collect all the PFElemnets of pfc which formed form eithr ECAL or HCAL.
%% then from each element we get the corresponding cluster reference & associated "hitAndFractions" = RecHits => input in ML algo.
% what are the input features?
%% Rechit position (x vs y) (z, Energy)

% note: for diagnostic checks we compare total rechit energy
% Raw energy (total raw pf)  , Gen level energy (total true pf) - X axis
% ?? - Y axis (slide 13) ?? frequency ?
% ** still don't understand plots in slide 14 ( is it corrected vs True? , color? frequency?)

% Summary
%........
%% Hadron Calibration is challenging.
%% The used input for ML modle is Rechits which are a PF element.
%% Calibration for (charged PF hadrons) using DRN & Chi squar method is done.
%% more improvement in the response for EH tha  H. 

% source: PF hadron cluster calibration 26/2/2024
%================================================

% What are the technical details of the calibration data,
%........................................................
%% centrally generated (true, raw) and reconstructed single pion (- charged hadron) samples.
%% In CMSSW version 12_6_4 (We used) , GT (global target) 126X_mcRun3_2023_forPU65_v4 (after ECAL corrections)
%% E range (2-200) (200-500) GeV (FlatRandomEGunProducer ) + |eta|<3  & |Phi| < 3.14 range 
%%%*This Particle Gun will generate one or more particles,
%%%all coming from the same vertex (0,0,0) and uniformly distributed in the user-specified phi-, eta- and energy-range*
% source: https://twiki.cern.ch/twiki/bin/view/CMSPublic/SWGuideParticleGuns

% How calibration using chi squre method is done?
%................................................

% How is it done using DRN?
%.........................

% what is the motivation for offline calibration?
%...............................................
% when plotting the resolution vs true energy for EH hadrons we can see that
%% the resolution is non linear
%%%* A "non-linear resolution" in calorimetry calibration refers to a method where the calibration constants applied to a calorimeter's signal
%%% are not a simple linear function of energy, meaning the relationship between the measured signal and the actual deposited energy is not a straight line,
%%% but instead exhibits curvature, requiring a more complex mathematical model to accurately convert signals to energy values;
%%% this is often necessary to account for inherent non-linearities in the calorimeter response,
%%% like variations in shower development at different energy levels or non-uniformities within the detector itself. (AI answer google)

%%%*a "calorimeter response" refers to the measurable signal produced by a calorimeter when a particle is completely absorbed within it,
%%% essentially representing the energy of the incident particle as a detectable output,
%%% typically in the form of light or electrical charge generated by the secondary particles produced in the particle shower;
%%% it is essentially how much signal is generated per unit of energy deposited in the calorimeter, and can vary depending on the type of particle being absorbed.*

% => check more general info about calorimeters see talk Calorimeters Energy measurement

% using  Chi square method, what are the energy/ Pseudorapidity dependent calibrationss?
% .........................................................................
%%%* In statistics, minimum chi-square estimation is a method of estimation of unobserved quantities based on observed data.
% => source: https://en.wikipedia.org/wiki/Minimum_chi-square_estimation
%% Energy dependent calibration: slide 5 
%% EH hadrons: E(corrected):a* E(raw Ecal) + b*E(raw Hcal) + O(EH) 
%% H hadrons: E(corrected): c*E(raw Hcal) + O(H)
%%% * coeffiecent O: accounts for the energy lost beacuse of the energy thresholds of the clustering algorithm.
%%% * coeffiecent a,b,c could be estimated using a large sample of simulated particles
%%% after getting the coeffiecents we can apply the corrections on the raw ECAL & HCAL energies. 


% types of calibration?
%.....................
% => check pg 47 Source: calorimetry for particle physics

%%%* Absolute calibration of a calorimeter is a process that determines the constants needed to convert raw ADC codes into energy depositions
%%% in the calorimeter counters. The goal is to ensure accurate measurements of heat changes in chemical reactions or physical processes. (AI answer google)


%%%%%%%%%%%%%%%%%%%%%%%%%%%%%%%%%%%%%%%%%%%%%%%%%%% 
\section{HCAL Cluster Calibration using GNN}

Hadronic showers in the CMS detector have both electromagnetic and hadronic components.

These showers are not dully contained in the ECAL but extend to HCAL.

The detector response is different for EM and HAD componenets which lead to non linear energy response.

The reconstructed energy of hadrons is the sum if all reconstructed hits (offline, PF reconstruct the RAW data) from the ECAL and HCAL.

For a given bin of ture energy (PF also reconstruct generated data or MC data):

by fitting the distribution of total RAW energy with Gaussian we obtain mu and std deviation then use them to get:

Resolution (std/mu) : which is a measure of accuracy

and Response [(mu/E true) -1] : which is measurement of precision. (we can see that is not linear)

Energy Reconstruction using conventional method (chi square) vs DRN (based on GNN).

The work in this thesis is done for Run3 data. 


\subsection{Strategy}


\subsection{Samples used and Training Details}

Include DRN architecture overview here. We could include the Training target.  

DRN training parameters. this include: Inpute layers (3) .. etc.

Sample details: centrally produced GEN-SIM-RAW (two momentum range) and privately reconstructed in CMSSW.

Rechits used as input. (we could expand more on this detail)

\subsection{Validation}

80\% of the dataset used for taining and 20\% for validation.  

% source: Chirayu talk 26/2 % 
%%%%%%%%%%%%%%%%%%%%%%%%%%%%%%%%%%%%%%%%%%%%%%%%%%% 
\section{Results and Discussion}

we present the results of response and resolution (from DRN vs Chi2) in  both Barrel region and endcap region.  
  \subsection{EH Hadrons}
  \subsection{H hadrons} 

%%%%%%%%%%%%%%%%%%%%%%%%%%%%%%%%%%%%%%%%%%%%%%%%%%%
  % side notes: reviewing other materials

  % How is physics analysis is done at the LHC?
  %% to study particles like Higgs boson, we need to be able to reconstruct such an event from inside the detector.
  %% this means we need to reconstruct the possible final products of Higgs decay: photon & pair of muon (decay products of Z)
  %% other things we need to reconstruct to do other analysis: electrons, hadrons.

  % How can we reconstruct collision events from the detector signals?
  %% By using Particle flow algorithm we can get an event description.
  %% PF uses info from each cms sub detector to identify and reconstruct the final products that have interacted with the detector. (5 types of particles)
  %% PF uses different algorithms to first reconstruct the elements: tracks, clusters and then to link them to form the PF blocks.
  %% from these blocks we can get the list of the PF particle candidates that could be analyized further
  %to get Hight level objects that should look similar to the event initial products ex: jets, MET.

  % How hadrons deposit energy in the CMS?
  % Why the calibration of hadrons clusters is important?
  % What methods are used in calibrating the energy of hadrons clusters?
  % What architecture used in GNN?
  % How can we imporve the GNN method?
  % How can we validate the results of the GNN training?

  % summary:
  %% this work shows the promising imporvment in hadron cluster calibration by using GNN.
  %% Note: check the PF meetings to see the latest updates related to this work 2023-2024
  
  %=====================================================

  % hadronic showers?
  %% First hadron calorimetry was used in studying the cosmic ray spectrum 1950s.
  %% Hadronic showers are more complicated than EM showers. they are deeper and they extend to HCAL. 
  %% Large number of nuclear processes involved in generating these showers.
  %% HAD showers have electromagentic and hadronic components.
  %% some hadrons start showering in ECAL while others in HCAL.

  % How the energy of the hadron is estimated?
  %% to be continue: source chirayu talk

  
























%%%%%%%%%%%%%%%%%%%%%%%%%%%%%%%%%%%%%%%%%%%%%%%%%%%
%\section{Captions}
%Figures are funny. They are notably different from tables and therefore requires some explanation. Here are the rules:
%\begin{enumerate}
%\item The caption is below the image
%\item The caption is centered if it's short, ie one line. 
%\item But if the text spans multiple lines, then it's left-justified. 
%\end{enumerate}

%Figure~\ref{figure_example1} has a short caption, and
%Figure~\ref{figure_example2} has a longer caption, demonstrating the required
%single-spacing.

%\begin{figure}[ht]
%\centering
%\includegraphics[width=1in]{baylor}
%\caption{This is a caption for this figure}
%\label{figure_example1}
%\end{figure}

%Figures like tables are floats and can be positioned anywhere. 
%This author favors placing images at the top. 
%But to illustrate, figure~\ref{figure_example2} is placed at the bottom. 

%\begin{figure}[b]
%\centering
%\includegraphics[width=0.2\textwidth]{baylor}
%\caption[The short table of contents version]{An example of a longer figure
%caption that spans multiple lines and has a corresponding short version for the
%table of contents.}
%\label{figure_example2}
%\end{figure}

%\section{Tables}

%Table~\ref{table_fruit} and Table~\ref{table_silly} demonstrate tables. Table
%captions differ slightly from figure captions, in that they are \textit{always}
%supposed to be centered, even if they use multiple lines.

%\begin{table}[h]
%\centering
%\caption[Fruits by color]{\centering Fruits listed by their color. Note that
%captions differ from figure captions in that they are \textit{always} supposed
%to be centered, even if they use multiple lines.}
%\label{table_fruit}
%\begin{tabular}{rl}
%    \hline
%    Fruit & Color  \\  \hline
%    Orange & Orange \\
%    Blue & Blueberry \\
%    Red & Cherry \\
%    Green & Apple \\
%    Yellow & Banana \\
%    Purple & Eggplant \\
%    \hline
%\end{tabular}
%\end{table}

%Tables can be anywhere in the text. They should be referred to \textbf{before} they make an appearance. 
%Tables can be placed between text, top of page, or bottom of page. This author personally prefers bottom of page. 
%There has to be triple space before the table captions, double space between caption and table, and triple space after the table. 

%The intext table (table~\ref{table_silly}) might look like it has more space after the table and before the text
%But that's just because the last item on the table is a horizontal line. 

%At the bottom of this page, is an example of a table set to [b]. This demonstrates the prettiness available to us by using tables at the bottom. 
%Bottom tables rock. As do top tables. 
%\begin{table}[b]
%\centering
%\caption[A bottom table]{A botom table illustrated}
%\label{table_silly}
%\begin{tabular}{ccc}
%    \hline
%    A & B & C \\  \hline
%    1 & 2 & 3 \\
%    4 & 5 & 6 \\
%    7 & 8 & 9 \\
%    \hline
%\end{tabular}
%\end{table}

%Nam dui ligula, fringilla a, euismod sodales, sollicitudin vel, wisi. Morbi auc-
%tor lorem non justo. Nam lacus libero, pretium at, lobortis vitae, ultricies et, tellus.
%Donec aliquet, tortor sed accumsan bibendum, erat ligula aliquet magna, vitae ornare
%odio metus a mi. Morbi ac orci et nisl hendrerit mollis. Suspendisse ut massa. Cras
%nec ante. Pellentesque a nulla. Cum sociis natoque penatibus et magnis dis par-
%turient montes, nascetur ridiculus mus. Aliquam tincidunt urna. Nulla ullamcorper
%vestibulum turpis. Pellentesque cursus luctus mauris.

%\begin{table}[!t]
%  \centering
%  \caption{This is a Top Positioned Table}
%  \begin{tabular}{ l c c c c }
%    \hline
%    \multirow{2}{*}{Interface} &
%    \multicolumn{2}{c}{Completion Time} &
%    \multicolumn{2}{c}{Throughput} \\
    
%    {} & Mean & Stdev & Mean & Stdev \\ 
%    \hline
%    Mouse only & 74s & 5.19 & 4.33bps & 0.35 \\
%    Mouse \& speech & 114s & 9.74 & 4.58bps & 0.46 \\
%    Gestures only & 116s & 14.76 & 2.66bps & 0.40\\
%    Gestures \& Speech & 136s & 14.82 & 2.83bps & 0.49\\
%    \hline
%  \end{tabular}
%  \label{tab:ranking}
%\end{table}


%Nam dui ligula, fringilla a, euismod sodales, sollicitudin vel, wisi. Morbi auc-
%tor lorem non justo. Nam lacus libero, pretium at, lobortis vitae, ultricies et, tellus.
%Donec aliquet, tortor sed accumsan bibendum, erat ligula aliquet magna, vitae ornare
%odio metus a mi. Morbi ac orci et nisl hendrerit mollis. Suspendisse ut massa. Cras
%nec ante. Pellentesque a nulla. Cum sociis natoque penatibus et magnis dis par-
%turient montes, nascetur ridiculus mus. Aliquam tincidunt urna. Nulla ullamcorper
%vestibulum turpis. Pellentesque cursus luctus mauris.
