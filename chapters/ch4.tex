%ch4 outline

Intro PF hadron cluster calibration using: chi squar method.

GNN intro 

PF hadron cluster calibration usin: Dynamic Reduction Network (DRN)

Taining deatails for DRN 


%plots: for EH , H hadrons.
%1- response vs trueE (covering 3 ranges of eta) 
%2- resolution vs trueE (covering 3 ranges of eta) 

for EH hadrons

\begin{figure}[ht]
%\centering
%\includegraphics[width=1in]{baylor}
\caption{EH - response vs trueE }
%\label{figure_example1}
\end{figure}

\begin{figure}[ht]
%\centering  
%\includegraphics[width=1in]{baylor}
\caption{EH - resolution vs trueE }
%\label{figure_example1}
\end{figure}


for H hadrons 

\begin{figure}[ht]
%\centering                                                                                                                                                                                                                                                                     
%\includegraphics[width=1in]{baylor}
\caption{EH - response vs trueE }
%\label{figure_example1}                                                                                                                                                                                                                                                        
\end{figure}

\begin{figure}[ht]
%\centering  
%\includegraphics[width=1in]{baylor}
\caption{E - resolution vs trueE }
%\label{figure_example1}
\end{figure}~

%%%%%%%%%%%%%%%%%%%%%%%%%%%%%%%%%%%%%%%%%%%%%%%%%%%
\section{Captions}
Figures are funny. They are notably different from tables and therefore requires some explanation. Here are the rules:
\begin{enumerate}
\item The caption is below the image
\item The caption is centered if it's short, ie one line. 
\item But if the text spans multiple lines, then it's left-justified. 
\end{enumerate}

Figure~\ref{figure_example1} has a short caption, and
Figure~\ref{figure_example2} has a longer caption, demonstrating the required
single-spacing.

\begin{figure}[ht]
\centering
\includegraphics[width=1in]{baylor}
\caption{This is a caption for this figure}
\label{figure_example1}
\end{figure}

Figures like tables are floats and can be positioned anywhere. 
This author favors placing images at the top. 
But to illustrate, figure~\ref{figure_example2} is placed at the bottom. 

\begin{figure}[b]
\centering
\includegraphics[width=0.2\textwidth]{baylor}
\caption[The short table of contents version]{An example of a longer figure
caption that spans multiple lines and has a corresponding short version for the
table of contents.}
\label{figure_example2}
\end{figure}

\section{Tables}

Table~\ref{table_fruit} and Table~\ref{table_silly} demonstrate tables. Table
captions differ slightly from figure captions, in that they are \textit{always}
supposed to be centered, even if they use multiple lines.

\begin{table}[h]
\centering
\caption[Fruits by color]{\centering Fruits listed by their color. Note that
captions differ from figure captions in that they are \textit{always} supposed
to be centered, even if they use multiple lines.}
\label{table_fruit}
\begin{tabular}{rl}
    \hline
    Fruit & Color  \\  \hline
    Orange & Orange \\
    Blue & Blueberry \\
    Red & Cherry \\
    Green & Apple \\
    Yellow & Banana \\
    Purple & Eggplant \\
    \hline
\end{tabular}
\end{table}

Tables can be anywhere in the text. They should be referred to \textbf{before} they make an appearance. 
Tables can be placed between text, top of page, or bottom of page. This author personally prefers bottom of page. 
There has to be triple space before the table captions, double space between caption and table, and triple space after the table. 

The intext table (table~\ref{table_silly}) might look like it has more space after the table and before the text
But that's just because the last item on the table is a horizontal line. 

At the bottom of this page, is an example of a table set to [b]. This demonstrates the prettiness available to us by using tables at the bottom. 
Bottom tables rock. As do top tables. 
\begin{table}[b]
\centering
\caption[A bottom table]{A botom table illustrated}
\label{table_silly}
\begin{tabular}{ccc}
    \hline
    A & B & C \\  \hline
    1 & 2 & 3 \\
    4 & 5 & 6 \\
    7 & 8 & 9 \\
    \hline
\end{tabular}
\end{table}

Nam dui ligula, fringilla a, euismod sodales, sollicitudin vel, wisi. Morbi auc-
tor lorem non justo. Nam lacus libero, pretium at, lobortis vitae, ultricies et, tellus.
Donec aliquet, tortor sed accumsan bibendum, erat ligula aliquet magna, vitae ornare
odio metus a mi. Morbi ac orci et nisl hendrerit mollis. Suspendisse ut massa. Cras
nec ante. Pellentesque a nulla. Cum sociis natoque penatibus et magnis dis par-
turient montes, nascetur ridiculus mus. Aliquam tincidunt urna. Nulla ullamcorper
vestibulum turpis. Pellentesque cursus luctus mauris.

\begin{table}[!t]
  \centering
  \caption{This is a Top Positioned Table}
  \begin{tabular}{ l c c c c }
    \hline
    \multirow{2}{*}{Interface} &
    \multicolumn{2}{c}{Completion Time} &
    \multicolumn{2}{c}{Throughput} \\
    
    {} & Mean & Stdev & Mean & Stdev \\ 
    \hline
    Mouse only & 74s & 5.19 & 4.33bps & 0.35 \\
    Mouse \& speech & 114s & 9.74 & 4.58bps & 0.46 \\
    Gestures only & 116s & 14.76 & 2.66bps & 0.40\\
    Gestures \& Speech & 136s & 14.82 & 2.83bps & 0.49\\
    \hline
  \end{tabular}
  \label{tab:ranking}
\end{table}


Nam dui ligula, fringilla a, euismod sodales, sollicitudin vel, wisi. Morbi auc-
tor lorem non justo. Nam lacus libero, pretium at, lobortis vitae, ultricies et, tellus.
Donec aliquet, tortor sed accumsan bibendum, erat ligula aliquet magna, vitae ornare
odio metus a mi. Morbi ac orci et nisl hendrerit mollis. Suspendisse ut massa. Cras
nec ante. Pellentesque a nulla. Cum sociis natoque penatibus et magnis dis par-
turient montes, nascetur ridiculus mus. Aliquam tincidunt urna. Nulla ullamcorper
vestibulum turpis. Pellentesque cursus luctus mauris.
