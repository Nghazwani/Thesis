\section{Particle Flow Reconstruction}

Particle Flow is an advance algorithm that combines all the information gathered using all the detectors in the CMS to reconstruct and identify final state particles. These initial  reconstructed particles are known as PF candidates are electrons, photons, charged and neutral hadrons, and muons.

The PF uses the basic elements: first element is the tracks information coming from the inner tracker, and muon chamber tracks. The other element is energy deposits information from ECAL and the HCAL. In this way the PF could give a better description for the collision event.

The particle flow is done in stages : first stage Clustering calorimeter energy deposits, then Linking all the tracks and clusters based on spatial a proximity (cells are neighboring each other in eta -phi view). Where Links can form between : Tracks \& ECAL clusters, Tracks \& HCAL clusters, ECAL \& HCAL clusters, Inner tracks and muon tracks, Muon tracks \& ECAL clusters, Muon tracks \& HCAL clusters.

(Explain PF blocks and how to get  form that PF candidates)


\section{Particle Flow Reconstruction}
\section{Calorimeter Cluster Calibration}
This is another section.
There is likewise triple space between the preceding text and this section's title.
The observant reader will notice that there is a double space between the section's title and this text.

\subsection{Calorimeter Cluster Calibration}
This is a subsection.
Just like the section, there is a triple space between the preceding text and the subsection's title.
And just like the section, there is a double space between the subsection's title and the subsection's text.

\subsection{More Subsections}
Same concept. Triple Space before title and double space after.
Just To show a demonstration, some garbage text will follow this in a new paragraph.

\lipsum[1]

\subsubsection{Subsubsections}
Subsubsections can be used. Maps to Level 5 on the graduate school requirements.
And just like section and subsection, there is a triple space before the title.
But then everything changes.
There is a period after the title followed by 2 spaces.
But fear not, this behavior is defined in the template.

\subsubsection{Capitalization is not the same}
The grad school asked to use sentence cases instead of header cases in the title of the subsubsection.
Alvin believes this will change eventually.

\section{Stacked Headings}

% \stack % DO NOT USE THIS COMMAND

\subsection{This is a Stacked Heading}
In other words there is no text in the section, and we immediately introduce a subsection.
To get spacing right when sections are adjacent to subsections (when headings are stacked),
we once needed to use the \texttt{$\backslash$stack} command in between.
But this is now handled correctly byt the teplate (.cls) file. So do not use the stack command.
It is marked obsolete in the template.

In other words: there has to always be a triple space between the text preceding the subsection title and the subsection title.
In this particular case the text preceding the subsection title is the section title.
But the same rule applies: triple space.
Assuming this changes, adjust accordingly in the template.

\subsection{Deeper Stacks}

% \stack % DO NOT USE THIS COMMAND

\subsubsection{This is a Stacked Deep Heading}
Same as before, there is no text between the subsection above and the subsubsection here.
But fear not. 'Tis taken care of by the template.
Unless the requirements change, in which case I'm adjusting the 1em vspace in the template should be fine.

\subsubsection{But This is Not}
But it's fine. It's just another Level 5. Or if you prefer -- a subsubsection. Whichever you choose to call it.
