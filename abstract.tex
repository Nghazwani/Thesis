
The Electromagnetic Calorimeter (ECAL) and Hadronic Calorimeter (HCAL) are key components of the CMS detector.
The ECAL is designed to measure the energies of electrons and photons, while the HCAL measures the energies of charged and neutral hadrons.
An algorithm called Particle Flow (PF) integrates information from various CMS sub-detectors to reconstruct and identify all particles produced in proton collisions.
Photons and neural hadrons are reconstructed using PF element energy clusters.
A proper calibration enhances particle identification and reduces the likelihood of misreconstructed energy excess.
%An accurate calibration of calorimeter response to Photons and neural hadrons is important to maximize the probability to identify these particles and minimize the rate of the misreconstructed energy excess.
Machine learning techniques, such as Boosted Decision Trees (BDT) and Graph Neural Networks (GNN), are employed to calibrate PF energy clusters, improving both the response and the resolution of the measured energy. 
%Machine learning algorithms like Boosted Decision Tree (BDT) and Graph Neural Network (GNN) are used to calibrate the PF energy clusters.These methods improve the response and resolutions of the measured energy by the calorimeters.
%=> note: add meaning of the response & resolution.
In this thesis, BDT is applied to calibrate PF ECAL clusters, while GNN is tested for hadronic cluster calibration. %BDT is used for PF ECAL Cluster Calibration, while GNN is used for Hadronic Cluster Calibration. (needs more details on the used data sets in performing both calibration and findings of each work)




%%%% draft 1 %%%%
%In this thesis BDT used for ECAL Cluster Calibration.
%in general the existing calibration derived from 2022 samples seems to continue working well.
% looked at difference between the HLT (online) vs offline ECAL PF clusters.used Run3 sample
% Also GNN used for Hadronic Cluster Calibration using +add conclusion+

