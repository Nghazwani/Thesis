

ECAL and HCAl are sub-detectors in the CMS experiments.
The ECAL is designed to measure the energies of electrons and photons, and the HCAL measures the energy of hadrons.
An algorithm called particle flow uses the CMS detector to reconstruct and identify all particles produced by the collision of protons.
Particle flow clusters in the calorimeters need to be calibrated to count missing energy.
Machine learning algorithms like BDT & GNN are used to calibrate the PF clusters.
These methods improve the response and resolutions of energies measured by the calorimeters.
In this thesis, BDT is used for ECAL Cluster Calibration, while GNN is used for Hadronic Cluster Calibration. 
(add few sentences about the findings of each work) 














%ECAL and  HCAl are one of the sub detectors of the cms experiments.

%The ECAL is designed to measure the energies of electrons and photons.
%The HCAL measures the energy of “hadrons”
    
%an algorithm called Particle Flow is used for  CMS detector to reconstruct and identify all produced particle.

%Clustering is done in the ECALand in HCAL
%PF clusters need to be calibrated to count of missing energy due to tracker material, gaps, dead channels etc.
%ML algorithms like : BDT & GNN are used in calibrating the PF clusters.
%These ML methods improve the response, resolutions of the measured energy.
    
%In this thesis BDT used for ECAL Cluster Calibration. in general the existing calibration derived from 2022 samples seems to continue working well.
%looked at difference between the HLT (online) vs offline ECAL PF clusters.used Run3 sample
    
%Also GNN used for Hadronic Cluster Calibration using +add conclusion+

