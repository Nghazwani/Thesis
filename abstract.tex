

The Electromagnetic Calorimeter (ECAL) and Hadronic Calorimeter (HCAL) are part of the CMS detector. The ECAL is designed to measure the energies of electrons and photons, while the HCAL measures the energies of charged and neutral hadrons.
An algorithm called Particle Flow (PF) combines different information from the CMS sub-detectors to reconstruct and identify all particles coming out of protons collisions. Photons and neural hadrons are reconstructed using the PF element energy cluster.
an accurate calibration of calorimeter response to Photons and neural hadrons is important to maximize the probability to identify these particles and minimize the rate of the misreconstructed energy excess.
Machine learning algorithms like Boosted Decision Tree (BDT) and Graph Neural Network (GNN) are used to calibrate the PF energy clusters.
These methods improve the response and resolutions of the measured energy by the calorimeters.
In this thesis, BDT is used for PF ECAL Cluster Calibration, while GNN is used for Hadronic Cluster Calibration.
(needs more details on the used data sets in performing both calibration and findings of each work) 

%%%% draft 1 %%%%
%ECAL and  HCAl are one of the sub detectors of the cms experiments.
%The ECAL is designed to measure the energies of electrons and photons.
%The HCAL measures the energy of “hadrons”
%an algorithm called Particle Flow is used for  CMS detector to reconstruct and identify all produced particle.
%Clustering is done in the ECALand in HCAL
%PF clusters need to be calibrated to count of missing energy due to tracker material, gaps, dead channels etc.
%ML algorithms like : BDT & GNN are used in calibrating the PF clusters.
%These ML methods improve the response, resolutions of the measured energy.
%In this thesis BDT used for ECAL Cluster Calibration. in general the existing calibration derived from 2022 samples seems to continue working well.
%looked at difference between the HLT (online) vs offline ECAL PF clusters.used Run3 sample
%Also GNN used for Hadronic Cluster Calibration using +add conclusion+

